
\documentclass[11pt, a4paper, titlepage]{article}

\usepackage[czech]{babel}
\usepackage[utf8]{inputenc}
\usepackage[T1]{fontenc}
\usepackage{times}
\usepackage[left=2cm,text={17cm,24cm},top=3cm]{geometry}

\newcommand{\textttsp}[1]{\texttt{ #1 }} % mezery kolem \texttt
\newcommand{\bibtex}{\textsc{Bib}\negthinspace\TeX} % BibTeX logo

\begin{document}
    \begin{titlepage}
        \renewcommand{\baselinestretch}{0.5}
        \thispagestyle{empty}
        \begin{center}
            \Huge\textsc{Vysoké učení technické v~Brně}\\\medskip
            \huge\textsc{Fakulta informačních technologií}\\
            \vspace{\stretch{0.382}} % zlaty rez
            \LARGE Typografie a~publikování\,--\,4.\,projekt\\\medskip
            \Huge Bibliografické citace\\
            \vspace{\stretch{0.618}}
        \end{center}
        \Large \today\hfill Marek Šipoš
    \end{titlepage}
    
    \newpage
    \setcounter{page}{1}
    
    \section{Typografie věda je}
    {\em\uv{Typografie je umělecko-technický obor, který se zabývá tiskovým písmem.}} Takto stručně a~výstižně definuje typografii Česká Wikipedie \cite{Wikipedia:Typografie}.
    Můžeme však něco, co provázelo lidstvo již od raných dějin (dokonce typografii někteří autoři definují jako {\em starobylé řemeslo} \cite{Bringhurst:The_Elements_of_Typographic_Style}) \uv{zaškatulkovat} jako pouhý obor? Nikoliv. Lidstvo si mělo potřebu nějak předávat nebo uchovávat informace už od nepaměti. Ať už rytím na stěny jeskyní nebo klínopisem.
    Protože se typografie zabývá (nejen) umístěním znaků a~rozmanitosti písem \cite{Ambrose:Typografie_graficky_design}, je velmi důležitým poslem a~průvodcem ve světě publikování. V~typografických systémech se používají různé znakové sady a~spousta druhů písem. Studium tohoto odvětví typografie je předmětem spousty prací a~publikací \cite{Cerny:Znakove_sady_v_typografickych_systemech}.
    \section{\LaTeX}
    V~moderní době informačních technologií ovšem potřebujeme jednoduchý způsob, kterým bychom mohli typografii efektivně využít. K~tomu slouží typografický systém \TeX\space a~jeho nadstavba \LaTeX, což je vlastně balík maker tohoto systému \cite{Wikibooks:LaTeX}.
    Tento systém se na akademické půdě využívá nejvíce při psaní závěrečných prací, dokumentací a~různých zpráv a~reportů. K~sázení textu je možné využít i~různá komerční řešení, \LaTeX\space je však zdarma a~poskytuje velmi kvalitní výstup \cite{FEKT:LaTeX}. Studenti z~Brna mají navíc štěstí! Aby se při sázení dokumentů orientovali, mají možnost se účastnit kurzu Typografie a~publikování, který byl před pár lety otevřen \cite{Krena:Zavedeni_predmetu_typografie_a_publikovani}. 
    \section{\bibtex\space podle normy ČSN ISO 690}
    Když už hovoříme o~využití \LaTeX u v~akademickém prostředí, součástí všech odborných textů musí být jistě správné bibliografické citace. K~tomu slouží nástroj \bibtex, který zjednodušuje jejich tvorbu a~správné formátování v~publikaci. V~České republice se v~dnešní době používá norma ČSN ISO 690. Problém je, že \bibtex\space tuto formu nedodržuje. Řešením může být použití stylu \textttsp{czplain} a~zahrnutí souboru stylu {\em *.bst} spolu se zdrojovým textem práce. Tento styl je produktem bakalářské práce \cite{Pysny:BIBTEX_STYL_PRO_CSN_ISO_690_A_CSN_ISO_690_2} a~je použit i~v~tomto dokumentu.
    \section{A~jak je to v~zahraničí?}
    Růst a~zkoumání typografie a~jejich pravidel se netýká jen České republiky. Všude ve světě se snaží umělci, vědci i~nadšenci prozkoumat veškerá zákoutí sázení textů. Zkoumají různé aspekty a~prvky, a~to i~tak dopodrobna, že jsou schopni vydávat články o~bílých znacích a~spojovnících \cite{Janssen:The_Rectangle_in_Typography}. Typografie se ovšem netýká jen formálního zápisu a~způsobu sázení. Má samozřejmě vliv i~na lidství jako takové, proto se v~zahraničí odborníci věnují i~analýze sociálním sémiotickým přístupem \cite{Leeuwen:Towards_a_semiotics_of_typography}.
    
    \newpage
    \renewcommand{\refname}{Literatura}
    \bibliography{proj4}
    \bibliographystyle{czplain}
\end{document}